\documentclass{article}

\usepackage{amsfonts,amsthm, amssymb, amsmath}
\usepackage{centernot}
\usepackage{authblk}
\usepackage{natbib}
\usepackage{graphicx}
\usepackage{color}
\usepackage{bm}

%\citestyle{nature}
\newtheorem{theorem}{Theorem}
\newtheorem{lemma}[theorem]{Lemma}
\newtheorem{proposition}[theorem]{Proposition}
\newtheorem{corollary}[theorem]{Corollary}
\newtheorem{defn}{Definition}
\newtheorem{block}{proposition}
\newcommand{\ord}{\text{ord}}

\usepackage[margin=0.75in]{geometry}


\begin{document}
\hfill \textbf{Cody Allen}

\hfill \textbf{\today}

\vspace{0.5cm}
%%%%%%%%%%%%%%%%%%%%%%%%%%%%%%%%%%%%%%
\begin{center}
{\huge \textbf{Minimizing Power Plant Cost - Simple Example}}
\end{center}

%\begin{figure}[h!]
%\centering
%  \includegraphics[width=7in]{hw1.png}
%\end{figure}

%\noindent
%\textcolor{red}{\large Final Grading: +11 points}
%\vspace{0.5cm}

\noindent
This example is \emph{Example 4.5} from \cite{EPS}.

PROBLEM:  Four generators are available to supply a power system peack load of 472.5 MW.  The cost of power, $C(P_i)$ from each generator, and maximum output, is given in (\$ U.S.) by:
\begin{align*}
C(P_1) &= 200 + 15P_1 + 0.20P_1^2 & & \text{Max. output 100 MW} \\
C(P_2) &= 300 + 17P_2 + 0.10P_2^2 & & \text{Max. output 120 MW} \\
C(P_3) &= 150 + 12P_3 + 0.15P_3^2 & & \text{Max. output 160 MW} \\
C(P_4) &= 500 + 2P_4 + 0.07P_4^2 & & \text{Max. output 200 MW} \\
\end{align*}
The spinning reserve is to be 10\% of the peak load and the transmission losses can be neglected.  Calculate the optimal loading of each generator.

\bigskip
\noindent
SOLUTION: The solution given in the book is given by graphical methods using marginal costs (derivatives of the above functions).  When a total of 347 MW is used, the optimal distribution is given as (in MW),
\begin{align*}
P_1^* &= 23 \\
P_2^* &= 64 \\
P_3^* &= 60 \\
P_4^* &= 200 \\
\end{align*}
In looking at the graphic used (Figure 4.16) to determine the solution provided, one concludes that the accuracy is within an order of magnitude.

The true solution can be found by using \emph{Quadratic Programming}.  Below we shall setup the problem, verify the requirements for optimal solution and then solve the problem.  The corresponding Matlab file provides this solution using the Sedumi and Yalmip solution.


Define,

\begin{align*}
A &= 
\begin{bmatrix}
	0.20 & 0 & 0 & 0  \\ 
	0 & 0.10 & 0 & 0  \\ 
	0 & 0 & 0.15 & 0  \\ 
	0 & 0 & 0 & 0.07  \\ 
\end{bmatrix}
& & &
b &= \begin{pmatrix}
15 \\ 17 \\ 12 \\ 2
\end{pmatrix}
& & &
c &= \begin{pmatrix}
1 \\ 1 \\ 1 \\ 1
\end{pmatrix}
& & &
m &= \begin{pmatrix}
100 \\ 120 \\ 160 \\ 200
\end{pmatrix}
& & &
d &= 200 +300+150+500 = 1150
\end{align*}

Then, the minimization problem can be stated as follows:

\begin{equation}\label{QP}
\begin{aligned}
& \underset{p}{\text{minimize}}
& & p^TAp + b^Tp + c \\
& \text{subject to}
& &  p \le m\\
&&& c^Tp = 347
\end{aligned}
\end{equation}

Now, we would like to ensure $p\ge 0$ as well, since we will not be producing negative power.  The inclusion of this constraint changes the problem.  However, if the solution returned by \eqref{QP} is all positive, there is nothing to worry about.

Note on inclusion of bound constraint:
%
%In certain QP problems, the lower bound of the solution may be inactive, meaning that $x^* > lb$, so that the optimal solution does not have any equalities on the lower bound constraint.  In which case, it need not be present in the constraints of the QP.  One way to check this is to calculate the minimum of the unconstrained QP, which is
%\begin{align*}
%x_{uc}^* = -\frac{1}{2}A^{-1}b
%\end{align*}
%where one can use the psuedo-inverse in place of the inverse if $A$ is singular.  If $ x_{uc}^* > lb$, then the solution to \eqref{QP} will not be affected by the lower bound constraint.
%
%In our case, 

\bibliographystyle{plain}
\bibliography{sample_bib}
\end{document}